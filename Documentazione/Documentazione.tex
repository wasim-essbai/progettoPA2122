\documentclass[
	a4paper,
	cleardoublepage=empty,
	headings=twolinechapter,
	numbers=autoenddot,
]{scrbook}

\usepackage{amsmath}
\usepackage{amsfonts}
\usepackage{amssymb}
\usepackage{import}
\usepackage{float}
\usepackage{cite}
\usepackage[signatures,swapnames,sans]{frontespizio}

\usepackage{todonotes}
\usepackage{verbatim}

\usepackage{color}
\usepackage{url} 

\newcommand{\Fig}[0]{Fig.}
\newcommand{\Eq}[0]{Eq.}

\pagestyle{plain}
\input{preamble}
\newcolumntype{C}{>{\centering\arraybackslash}p{3cm}}

\begin{document}
	\frontmatter
	
	\pdfbookmark{Title page}{titlepage}
	\begin{frontespizio}
		\Margini{3cm}{3cm}{3cm}{3cm}
		\Universita{Bergamo}
		\Logo[43.332mm]{ImageFiles/unibg-mark}
		\Divisione{Scuola di Ingegneria}
		\Corso[Laurea Magistrale]{Ingegneria Informatica}
		\Titolo{Sistema di prenotazioni lezioni}
		\Sottotitolo{Progetto del corso di Programmazione Avanzata}
		\Punteggiatura{}
		\Candidato[1060652]{Wasim Essbai}
		\Annoaccademico{2021--2022}
		\begin{Preambolo*}
			\usepackage[italian]{babel}
			\usepackage[T1]{fontenc}
			\usepackage[utf8]{inputenc}
			\usepackage{microtype}
			\usepackage{lmodern}
			\graphicspath{{img/}}
			
			\renewcommand{\frontinstitutionfont}{\fontsize{14}{17}\bfseries\scshape}
			\renewcommand{\fronttitlefont}{\fontsize{17}{21}\bfseries\scshape}
			\renewcommand{\frontfootfont}{\fontsize{12}{14}\bfseries\scshape}
		\end{Preambolo*}
	\end{frontespizio}
	
	\tableofcontents
	\listoffigures
	\mainmatter
	
	\chapter*{Introduzione}
	\addcontentsline{toc}{chapter}{Introduzione}
	Il progetto realizzato consiste in due programmi: Il primo sviluppato con il linguaggio C++, basato sul paradigma della programmazione ad oggetti, mentre il secondo in Haskell, seguendo il paradigma della programmazione funzionale. In seguito vengono descritti entrambi i programmi con le loro funzionalità.
	
	\chapter*{C++}
	\addcontentsline{toc}{chapter}{C++}
	Il programma scritto con il linguaggio C++ consiste in un sistema di gestione per la prenotazione di tutorati. Il software usa diversi costrutti del linguaggio come la coppia costruttore/distruttore, ereditarietà pubblica/private, modificatori di visibilità dei membri di una classe e smart pointers. Inoltre, si fa uso anche della libreria \textit{Standard Template Libray}, offerta dal linguaggio.
	\import{./TextFiles/C++/}{Features.tex}
	
\end{document}