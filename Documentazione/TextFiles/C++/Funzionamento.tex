\section{Funzionamento}
Di seguito viene descritto il funzionamento del programma e la sua struttura interna.
\subsection{Entità principali}
Le tre tipologie di lezione rappresentano le entità fondamentali del programma. In figura \Fig~\ref{fig:ClassDiagram} viene riportato il diagramma delle classi con descritte le interfacce. In particolare, è presente una classe base che è Lezione e che contiene i campi comuni a tutte le lezioni. Dopodiché ci sono le classi che rappresentano effettivamente le entità di interessa e che sono derivate da Lezione. In particolare, LezioneSingola e LezioneCollettiva eredita in modo pubblico da Lezione mentre RicevimentoAperto in modo private, questo perché quest'ultima non ha tutti gli attributi di Lezione, quindi anche non tutte le funzionalità e pertanto restringe la sua interfaccia. In fine, sono indicate anche le relazioni tra le classi e insieme alle molteplicità.
\begin{figure}[ht]
	\centering
	\includegraphics[width=0.7\linewidth]{ImageFiles/c++/ClassDiagram}
	\caption{Diagramma delle classi.}
	\label{fig:ClassDiagram}
\end{figure}

\subsection{Dinamica}
La dinamica del programma viene descritta in figura \Fig~\ref{fig:ActivityDiagram}. All'avvio vengono presentato il menu da cui l'utente può scegliere l'operazione da svolgere, la scelta viene fatta inserendo un numero, che corrisponde all'operazione. La descrizione nel diagramma è ad alto livello, ovvero viene riportata la dinamica globale senza dettagliare l'algoritmo che implementa le specifiche attività. 
Alla scelta dell'operazione consegue l'invocazione di una procedura che la soddisfa, previa controllo dell'input, ovvero che il numero scelto corrisponda effettivamente ad un'operazione.
\begin{figure}[ht]
	\centering
	\includegraphics[width=0.7\linewidth]{ImageFiles/c++/ActivityDiagram}
	\caption{Diagramma delle attività.}
	\label{fig:ActivityDiagram}
\end{figure}
Le procedure sono implementate in una classe dedicata chiamata Gestore. Questa non è rappresenta un'entità bensì un \textit{controller}, che quindi implementa ed offre le varie funzionalità. Queste procedure vengono invocate dal modulo \textit{main} del programma, dove infatti viene effettuata la scelta.

